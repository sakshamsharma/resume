% ----------------------------------------------------------------------------------------
% SECTION TITLE
% ----------------------------------------------------------------------------------------

\cvsection{Projects}

% ----------------------------------------------------------------------------------------
% SECTION CONTENT
% ----------------------------------------------------------------------------------------

\begin{cventries}

  % ------------------------------------------------

  \cventry
  {Undergraduate Project, Prof. Sandeep Shukla}
  {moVi: Mobile Video Chat Protocol
    \ \ \ \normalfont\href{https://github.com/netsecIITK/moVi}
    {github.com/netsecIITK/movi}}
  {IIT Kanpur}
  {Aug. 2016 - Present}
  {
    \begin{cvitems}
    \item Working on developing a protocol for video communication
      vis-a-vis \href{https://mosh.org/}{Mosh (mobile shell)}.
    \item Using UDP to set up a connection-less and secure channel,
      persistent across network IP and location changes.
    \item Trade-off between video quality, stuttering rate vs network
      resources and reliability required.
    \item Exploring, devising ways for transmitting video
      with minimum requirements, distortion with UDP packets.
    \end{cvitems}
  }

  \cventry
  {Course Project, Prof. Piyush Kurur and
    Prof. Satyadev Nandakumar}
  {Anonymous and private pair matching platform}
  {IIT Kanpur}
  {Oct. 2016 - Present}
  {
    \begin{cvitems}
      \item Designed and in the process of implementing an algorithm
        for fully anonymous matching of couples.
      \item Allows the end users to put zero trust in the server
        (admin), irrespective of code on the backend.
      \item Using homomorphic two party computation (assisted by an
        un-trusted server backend) and asymmetric encryption to
        ensure confidentiality and integrity.
    \end{cvitems}
  }

  \cventry
  {Undergraduate Project, Prof. Sandeep Shukla}
  {HOP: HTTP proxy for arbitrary protocols
    \ \ \ \normalfont{\href{https://github.com/sakshamsharma/HTTP-Over-Protocol}
      {acehack.org/hop}}}
  {IIT Kanpur}
  {Aug. 2016}
  {
    \begin{cvitems}
    \item A proxy to wrap and unwrap arbitrary binary data from/to
      HTTP packets, to bypass possible proxy restrictions.
    \item Stayed in \textbf{top 5} of \textit{Github's trending}
      repositories in C++ for 2 days.
    \end{cvitems}
  }

  \cventry
  {Member, Team Robocon IIT Kanpur, Prof. Bhaskardas Gupta}
  {ABU Robocon 2015, Badminton playing robots}
  {IIT Kanpur}
  {Oct. 2014 - Mar. 2015}
  {
    \begin{cvitems}
    \item Programmed and built 2 semi-autonomous robots
      capable of playing badminton on a standard size court.
    \item Used image processing with OpenCV to detect the shuttle
      and predict the trajectory.
    \item Used Kinect and Stereo Vision to get depth of
      field. Programmed the robot using Arduino run by Odroid.
    \item Finished 11th among 85 teams all over India.
    \end{cvitems}
  }

  \cventry
  {Course Project, Computer Networks, Prof. Sandeep Shukla}
  {Network Concepts Implementation}
  {IIT Kanpur}
  {Aug. 2016 - Present}
  {
    \begin{cvitems}
    \item Wrote an HTTP Server, an HTTP Proxy and an implementation of
      STCP Protocol using socket programming.
    \end{cvitems}
  }


  \cventry{Consecutive two time hackathon winner}
  {Microsoft code.fun.do hackathon}
  {IIT Kanpur}
  {Jan. 2015, Sept. 2015}
  {
    \begin{cvitems}
    \item An application to parse and plot graphs of implicit
      mathematical functions using C\#.
    \item A platform to learn coding, with a
      custom online judge written in Node.js. National 5th (coding milestone).
    \end{cvitems}
  }

  % ------------------------------------------------
\end{cventries}

%%% Local Variables:
%%% mode: latex
%%% TeX-master: "../resume_twopage"
%%% End:
