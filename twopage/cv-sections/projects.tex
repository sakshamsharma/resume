% ----------------------------------------------------------------------------------------
% SECTION TITLE
% ----------------------------------------------------------------------------------------

\cvsection{Projects}

% ----------------------------------------------------------------------------------------
% SECTION CONTENT
% ----------------------------------------------------------------------------------------

\begin{cventries}

  % ------------------------------------------------

  \cventry
  {Undergraduate Project, Prof. Sandeep Shukla}
  {\href{https://github.com/netsecIITK/moVi}{\entrytitlestyle{moVi: Mobile Video Chat Protocol}}
    \ \ \ \normalfont\href{https://github.com/netsecIITK/moVi}
    {github.com/netsecIITK/movi}}
  {IIT Kanpur}
  {Sept. 2016 - Present}
  {
    \begin{cvitems}
    \item Developed a client for video communication
      akin to \href{https://mosh.org/}{Mosh (mobile shell)}.
    \item Used UDP to set up a connection-less and secure channel,
      persistent across network IP and location changes.
    \item Dynamic trade-off between video quality, stuttering rate vs network
      resources and reliability required.
    \item Implemented State Synchronization Protocol, UDP Hole
      Punching, along with dynamic tweaking of video quality, to
      maintain performance and reliability across scenarios.
    \end{cvitems}
  }

  \cventry
  {Course Project, Prof. Piyush Kurur and
    Prof. Satyadev Nandakumar}
  {\href{https://github.com/pclubiitk/puppy-love}{\entrytitlestyle{Anonymous
        and private pair matching platform}
      \ \ \ \normalfont{\href{https://github.com/pclubiitk/puppy-love}
      {acehack.org/puppy}}}}
  {IIT Kanpur}
  {Oct. 2016 - Present}
  {
    \begin{cvitems}
      \item Designed and implemented an algorithm
        for fully anonymous matching of couples.
      \item Allows the end users to put zero trust in the server
        (admin), irrespective of code on the backend.
      \item Used secure two party computation (assisted by an
        un-trusted server backend), asymmetric encryption to ensure
        confidentiality and fairness even during matching.
    \end{cvitems}
  }

  \cventry
  {Undergraduate Project, Prof. Sandeep Shukla}
  {\href{https://github.com/sakshamsharma/HTTP-Over-Protocol}{\entrytitlestyle{HOP:
        HTTP proxy for arbitrary protocols}}
    \ \ \ \normalfont{\href{https://github.com/sakshamsharma/HTTP-Over-Protocol}
      {acehack.org/hop}}}
  {IIT Kanpur}
  {Aug. 2016}
  {
    \begin{cvitems}
    \item A proxy to wrap and unwrap arbitrary binary data from/to
      HTTP packets, to bypass possible proxy restrictions.
    \item Stayed in \textit{top 5} of \textit{Github's trending}
      repositories in C++ on first 2 days of release.
    \end{cvitems}
  }

  \cventry
  {Member, Team Robocon IIT Kanpur, Prof. Bhaskardas Gupta}
  {ABU Robocon 2015, Badminton playing robots}
  {IIT Kanpur}
  {Oct. 2014 - Mar. 2015}
  {
    \begin{cvitems}
    \item Programmed and built 2 semi-autonomous robots
      capable of playing badminton on a standard size court.
    \item Used image processing with OpenCV to detect the shuttle
      and predict the trajectory.
    \item Used Kinect and Stereo Vision to get depth of
      field. Programmed the robot using Arduino run by Odroid.
    \item Finished 11th among 85 teams all over India.
    \end{cvitems}
  }

  \cventry{Consecutive two time hackathon winner}
  {Microsoft code.fun.do hackathon}
  {IIT Kanpur}
  {Jan. 2015, Sept. 2015}
  {
    \begin{cvitems}
    \item An application to parse and plot graphs of implicit
      mathematical functions using C\#, for Windows Phone.
    \item A platform to learn coding for Windows Phone, with a
      custom online judge written in Node.js.
    \end{cvitems}
  }

  % ------------------------------------------------
\end{cventries}

%%% Local Variables:
%%% mode: latex
%%% TeX-master: "../resume_twopage"
%%% End:
